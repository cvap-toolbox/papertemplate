\section{Experiments}\label{experiments}

We present results that shows (i) How affordance classification is
enhanced by metric learning, (ii) How incremental learning affects
classification, (iii) How the learned transform can ground the
affordance in the feature representation and how that can be visualized
on the object, and (iv) How the classification and grounding is affected
by a human ordering of object similarity.

\subsection{Affordances \& Data
Collection}\label{affordances-data-collection}

We define the affordances to be reasonable from a human perspective but
also not to stretch too much from what is possible for the robot to read
from its sensors. For example we use a Kinect1 camera to collect data
which is quite sensitive and noisy both in 2D and depth, this limits the
resolution and measurements of the point cloud of the object. We choose
the following affordances

\small

\begin{itemize}
\tightlist
\item
  \textbf{Hangable} - All objects that affords hanging such as cups with
  ears.
\item
  \textbf{Putable} - All objects that affords containing something.
\item
  \textbf{Rollable} - All objects that affords rolling on the table such
  as cups without ears, bottles, etc.
\item
  \textbf{Lift Top} - All objects that has a top that affords removing
  such as bottles, pens, etc.
\item
  \textbf{Tool} - All objects that affords some kind of tool use this
  might be a spatula, hammer, screwdriver, etc.
\item
  \textbf{Brushing} - All objects that affords brushing.
\item
  \textbf{Hammering} - All objects that affords hammering this includes
  a hammer, a screwdriver, a brush, that is, objects that be replaced if
  a hammer is not available.
\item
  \textbf{Stirring} - All objects that affords stirring again spatulas,
  or other objects that possibly could be used for stirring.
\item
  \textbf{Spraying} - All objects that affords spraying, that is, spray
  bottles.
\item
  \textbf{Scraping} - All objects that affords some kind of scraping
  action such as window scraper, etc.
\item
  \textbf{Writeable} - All objects that affords writing, that is, pens.
\item
  \textbf{Drinking} - All objects that affords drinking from such as
  bottles, cups, etc. Here we define some objects as not drinkable from
  social conventions or direct harm, even though they technically
  affords it.
\item
  \textbf{Opening} - All objects that affords opening, that is all food
  item, boxes, bottles, etc.
\item
  \textbf{Squeezing} - All objects that affords squeezing something out
  them such as shampoo bottles, etc.
\item
  \textbf{Playing} - All objects that affords some kind of playing such
  as maracas, some balls and toys.
\end{itemize}

\normalsize

Our dataset consists of a set of 103 diverse instances, some are the
same object but from different viewpoints, such as different tools,
cups, bottles, balls, boxes, pots, cans, cleaning fluids, etc. Due to
space limitations a link is provided to all images and point-clouds of
the dataset.

We perform the collection of object data using a Kinect camera with
objects placed on a flat surface in front of the robot. The robot
observes and segments out the object and then records features over it.
Each object is associated with a binary label indicating if it affords
an action or not. In addition the robot might be provided with a human
k-NN ordering for each of the objects in the demonstrated set, where
\(k\) can vary for each object.

\subsection{Learning the Distance
Metric}\label{learning-the-distance-metric}

We preprocess the features by centering and scaling to unit variance. To
evaluate the learning we run the optimization a 100 times using random
splits of the collected data with a ratio of \(70/30\) training-test
data. Due to the non-convexity and number data points to parameters
ratio the LMCA is quite sensitive to initialization and overfits easily.
Cross-validation for parameter learning is thus best done using
leave-one-out which is quite costly even though we use a constraint
activation similar to \cite{Weinberger:2009to} to reduce the
computational cost.

A less expensive approach is to keep track of the ratio between the two
error terms and the k-NN leave-one-out error on the training dataset. A
low ratio and zero leave-one-out error almost always indicates
overfitting. As for the \(\beta\) parameter we can again think of the
analogy between \(D\) regression functions, since we have standardized
the dataset a good initial guess is \(\beta=D\).

\subsection{Affordance Classification}\label{affordance-classification}

Affordance classification accuracy and standard deviation can be found
in table \ref{fig:affordance_table1}. As we can see the accuracy is
above \(80\%\) for most of the affordances, LMCA outperforming or
scoring equal to k-NN on all affordances. The conclusion we can draw
from this is that classification is relatively simple when we have the
right representation and even more simple when exactly the right
features are selected. Further on, we recognize that

\subsubsection{Human Provided k-NN}\label{human-provided-k-nn}

LMCA can use any distance measure for computing the nearest neighbors
that are used in Eq.\ref{lossfun} to learn the metric. We use the
\(l_2\) distance for initialization, as that is what is used in the loss
function. Another approach, is to let a human demonstrator set the \(k\)
nearest neighbors for each object. This might lead to an ordering of the
neighbors that does not reflect the \(l_2\) distance but will provide
more valuable information into what features are relevant.

\subsubsection{Incremental Learning}\label{incremental-learning}

As mentioned in the introduction experiments in \cite{POSNER:1967ef} and
by others using distorted patterns have shown that the categorization
process for humans happens in a continuum. In the beginning individual
exemplars are remembered but as more examples are introduced a
generalization process takes place. We are therefore interested in how
the affordance transforms changes as more examples are introduced.

\begin{figure}
\captionsetup[subfigure]{labelformat=empty}
%
\subfloat[][]{
		\begin{tabular}{ l l l }
			\textbf{Affordance} & \textbf{k-NN} & \textbf{LMCA} \\ \hdashline[0.5pt/2pt]
			Hangable 	& $0.$ & $0.$ \\ \hdashline[0.5pt/2pt]
			Putable 	& $0.$ & $0.$ \\ \hdashline[0.5pt/2pt]
			Rollable 	& $0.$ & $0.$ \\ \hdashline[0.5pt/2pt]
			Lift Top 	& $0.$ & $0.$ \\ \hdashline[0.5pt/2pt]
			Tool 			& $0.$ & $0.$ \\ \hdashline[0.5pt/2pt]
			Brushing 	& $0.$ & $0.$ \\ \hdashline[0.5pt/2pt]
			Hammering & $0.$ & $0.$ \\ \hdashline[0.5pt/2pt]
			Stirring 	& $0.$ & $0.$ \\ \hdashline[0.5pt/2pt]
		\end{tabular}
}
%
%
\subfloat[][]{
		\begin{tabular}{ l l l }
			Spraying 	& $0.$ & $0.$ \\ \hdashline[0.5pt/2pt] 
			Scraping 	& $0.$ & $0.$ \\ \hdashline[0.5pt/2pt] 
			Writeable & $0.$ & $0.$ \\ \hdashline[0.5pt/2pt] 
			Drinking  & $0.$ & $0.$ \\ \hdashline[0.5pt/2pt] 
			Opening 	& $0.$ & $0.$ \\ \hdashline[0.5pt/2pt] 
			Squeezing & $0.$ & $0.$ \\ \hdashline[0.5pt/2pt] 
			Playing 	& $0.$ & $0.$ \\ \hdashline[0.5pt/2pt]
		\end{tabular}
}
%
	\caption{Affordance classification rate in \% k-NN vs. LMCA.}
	\label{fig:affordance_table1}
\end{figure}
