\section{Experiments}\label{experiments}

\subsection{Data Collection}\label{data-collection}

We perform the collection of object data using a Kinect camera with
objects placed on a flat surface in front of the robot. The robot
observes and segments out the object and then records features over it.
Each object is associated with a binary label indicating if it affords
an action or not. In addition the robot might be provided with a k-NN
ordering for each of the objects in the demonstrated set, where \(k\)
can vary for each object and \(k=1,2,3..\).

\subsection{Learning the Distance
Metric}\label{learning-the-distance-metric}

We preprocess the non-histogram features by centering and scaling to
unit variance. To evaluate the learning we run the optimization a 100
times using random splits of the collected data with a ratio of
\(70/30\) training-test data. The biggest drawback of the LMCA algorithm
is that the parameter learning is done by cross-validation. Due to the
non-convexity and small data point to parameter ratio the LMCA is quite
sensitive to initialization and overfits easily. Cross-validation thus
is best done using leave-one-out which is quite costly even though we
use a constraint activation similar to \cite{Weinberger:2009to} to
reduce computation.

A less expensive approach is to keep track of the ratio between the two
error terms and the k-NN leave-one-out error on the training dataset. A
high ratio and zero leave-one-out error almost always indicates
overfitting. As for the \(\beta\) parameter we can again think of the
analogy between \(D\) regression functions, since we have standardized
the dataset a good initial guess is \(\beta=D\).
