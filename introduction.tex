\section{Introduction}\label{introduction}

Imagine you are part of an experiment where you are shown a set of
patterns, and you are trained to recognize them. Each pattern is a
distorted version of some protoype pattern. Next you are shown a new set
of distorted patterns, and you are told to classify the patterns as seen
belonging to one of the prototypes. This is a classic experiment done by
Posner et. al \cite{POSNER:1967ef} to probe how humans learn to abstract
visual stimuli and form categories. The experiments show how categories
form in a continiuum rather than in distinct classes. The experiment
also show that sufficiently distinct members are treated as individual
exemplars rather belonging to some specicific category.

For a robotic agent acting in an unstructured environment learning
object affordances is similar to the pattern experiment - there is not
much need for object individuation other than when encountering unqiue
objects - the key is rather to understand the underlying common pattern
among the objects that form the category. To differentiate and to
categorize objects we need to learn how to measure the similarity to the
patterns we have previsouly observed and what in the pattern to measure
for different categories.

From a learning perspective this boils down to classification and
feature selection. Most approaches have focused on either of the two. In
\cite{Stoytchev:2005il,Chao:2011id,Niekum:us} the authors tries to learn
pre and post-conditions of the features that affects the outcomes from
applying the actions. In Yuruten et al. \cite{Yuruten:2013hr} the
authors learn a robot to match descpriptive nouns to object features
using feature selection with a radial based SVM.
\cite{Stark:2008bx}\ldots{}

From this perspective our approach is more holistic, we consider
learning feature selection and classification as an integral part of
each other. We realize firstly that the distance between objects that
are similar should be close in feature space. Feature dimensions are
high-dimensional and any distance measure will be hampered by all points
being close in high dimensions, due to this we would like to discount
features that are irrelevant for the categorization at hand. To this end
we employ a metric learning algorithm, that learns a transform that
pushes similar items together and non-similar far away. Since most
features are redundant for a specific category the transform can also be
forced to project onto a lower dimensional feature space. In addition we
introduce a regularizer in the learning algorithm that penalizes
transforms that are non-sparse and which induces feature selection.

Much research argue that learning affordances, should be done through
grounding in the sensorimotor system of the agent. Affordances
recognition and discovery is as much tied to ineraction with the object
as it is to visual stimuli. In Sahin et al. \cite{Sahin:2007gr} for
example, the authors define an agent centric definition of affordances
that relies of the effects of interaction with the world. Much of the
research on affordance learning follows this paradigm - that only by
forming its own representation through grounding in its own sensory
input will the agent be able to fully understand the affordance.

Most of the affordance learning through interaction have so far been
focused on simple affordances such as pushing, rolling, etc. For example
in Modayil et. al \cite{Modayil:2008it} an autonomous robot learns about
the world through random motor babbling, observing changes in the
perceived world and makes goal oriented plans based on matching the
learned actions to a given plan.

Learning for more complex tasks on the other hand, such as interaction
with everyday objects learning is more complicated. In robotics the
problem is usually approached by Learning from Demonstration (LfD) as
this is the way humans also learns complex tasks. Here again, learning
is usually tied to letting the robot match the observation to the action
through its own sensors, such as in \cite{Faria:2014ug}{[}Vet inte om
det är bra citering{]}.

One integral part in performing complex affordances are grasping and
manipulation, research have mostly focused on the grasping of objects
regardless of task context \cite{Bohg:H95zG3Ya}. However, steps have
been taken torwards learning task dependent grasp selection by LfD of
human priors on grasp placement \cite{Hjelm:2015hw}.

Understanding of how an object should be manipulated thus lays as much
in learning complex motor commands as recognizing how an object is best
manipulated. Understanding what in the feature representation of an
object makes the object afford the action is key to understanding this.
From this understanding it also follows that an agent can greatly reduce
the space of possible manipulation actions and gain a deeper
understanding of the affordance. Previous efforts have been part in
grasp planning models and focused on different forms of semantic
segementation. In Aleotti et al. \cite{Aleotti:2011hc} the authors
segments objects as graphs where each node can be linked to specific
grasp manipulations. Antanas et al. \cite{Antanas:2014uo} divides
objects into tools which have segments handles and usable areas, and
other objects which have segments that are bottom, middle and top of the
object as well as possible handles. The authors use these segementations
to compute probabilities of graspable parts. Similarly Faria et al.
\cite{Faria:2014ug} segments objects using the main axis of the object
and a clustering of the local curvature using a Gaussian Mixture Model
(GMM) which can be associated with particular types of grasps.

Our model breaks with these approaches by assuming that affordances of
objects can both be formulated around global features such as the
elongatedness of the object, the volume, etc. or local features such as
the specific color consistency across objects, single color caps for
examples, or the local curvature of mug handles.

Instead of tuning a segmentation algorithm to fit our purposes we want
the agent to be able to discover the commonn denominator in the feature
space of the objects sharing a specifc affordance. This means that the
semantic meaning of the affordance will be grounded in the sensory input
of the agent. This grounding will have two key components - the
transform of the feature space and the instances in the transformed
feature space.

As mentioned above we are not interested in object individuation as we
think this is a hinderance to truly understanding affordances. We
instead want to formulate the understanding around the feature
transform. This enables us to, in the case of a local feature, map the
affordance to a specific portion of the object. As a special case of
understanding common structures we show how we can learn a segmenation
that captures the important parts of the object. Further on, this makes
comparison between affordances possible by computing the distance
between the transforms.

As for the transformed instances they are used in this work to classify
object affordances using k-Nearest Neighbor (kNN) and to show how
objects are iterrelated in the transform spaced. In \cite{Hjelm:2015hw}
they were used to for grasp selection

\cite{Nikandrova:2015uu} \cite{Thomaz:2009uk} \cite{Griffith:2009cm}
\cite{Stark:2008bx}

\begin{comment}

ONE IDEA IS TO USE SEVERAL BOW REPRESENTATIONS AT DIFFERENT RESOLUTIONS!!



In the future we can expect that skill transfer will happen from robot to human, this will further the need for the agent in understandning for 

Skill-transfer from human to human works in the same manner as in the above experiment. We show some objects, explain the underlying invariant features, similarities and how the action works. The learner can then use the acquired skills to explore and act in the environment using the invariant knowledge to extrapolate and solve tasks. An important point is that that the transfer is not an exact copy due to the difference in semantic and sensory capabilities of the teacher and learner. 

In the future it is reasonable to assume that this skill-transfer will also happen between human and robots. Communication would be greatly enhanced by giving the robot the ability to ground the sensory input in semantic meaning, that is, find the the invariant features of the group and form metrics over these to discriminate which objects belongs to the group or not. Further on, utilizing this learning strategy as opposed to downloading an already trained classifier, would lead to greater adaptability and autonomy as the experience would be grounded in the robots own experience. 

If we go even further we can assume that in the future skill transfer will also happen from robot to human. Here semantic meaning grounded in sensory input  will be the only way to transfer knowledge as there is yet no known way of programming a human brain for learning complex tasks. 

Previous approaches to learning affordances have focused on....

Most of the above mentioned approaches focuses on learning affordances using large datasets that does not discover the invariant features of the object and where the focus has shifted from grounding to model accuracy. These models while useful for detection will not be very helpful in human-robot interaction. 




Lära eigen spectrumet dominanta plotta Renauds 

Hur annorlunda är representationen  för de olika tasken projektion av data punkter overlay. 
Vilka task är korrelerade? 

\end{comment}
