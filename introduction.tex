\section{Introduction}\label{introduction}

Imagine you are part of an experiment where you are shown a set of
patterns, where each pattern is a distorted version of some prototype
pattern. The researchers train you to recognize and categorize the
patterns. Next you are shown a new set of distorted patterns, and you
are told to classify the patterns as seen belonging to one of the
prototypes. This is a classic experiment done by Posner et. al
\cite{POSNER:1967ef} to understand how humans learn to abstract visual
stimuli and form categories. The experiment shows how categories form in
a continuum rather than in distinct classes and that sufficiently
distinct members, heavily distorted patterns, are treated as individual
exemplars rather than belonging to some specific category.

For a robotic agent acting in an unstructured environment learning
object affordances is similar to the pattern experiment --- there is not
much need for object individuation other than when interacting with
unique objects --- the key is rather to understand the underlying common
pattern among the objects that can be categorized under a specific
affordance. To differentiate and categorize objects the agent needs to
learn what and how to measure the similarity to the patterns it has
pervasively observed for the different categories.

From a learning perspective this boils down to feature selection and
classification. Most approaches have focused on the feature selection
formulated as pre- and post-conditions. In
\cite{Stoytchev:2005il, Chao:2011id, Niekum:us} the authors tries to
learn pre and post-conditions of the features that affects the outcomes
given the actions. While successful at learning and performing
manipulation they do not adddress more complex affordances involving
human made objects and their understanding is limited to the setting the
affordances was learned in.

From this perspective our approach is more general and holistic, we
consider learning the feature selection and classification as an
integral part of each other that can be learned irrespectively of
manipulation context. We realize that to use similarity for affordance
categorization, the distance between objects that are similar should be
close in the feature space. To this end we learn the similarity metric
from the data, that is, we learn a affordance specific transform that
pushes similar items together and non-similar items far away.

Any features with general discriminative capabilities are
high-dimensional and noisy, so that possible distance measures will be
distorted. However, we can observe that most features are redundant for
determining a specific affordance category, this implies that the
category specific features inhabits a lower dimensional space. We
therefore force the transform to project onto a lower dimensional space.
In addition, we introduce a regularizer in the learning algorithm that
penalizes transforms that are non-sparse and induce feature selection.
This means that we are simultaneously learning to categorize objects but
also what in the category makes it afford an action.

Most research on affordances argue that learning should be done through
grounding in the sensorimotor system of the agent and that only by
forming its own representation will the agent be able to fully
understand the affordance. Affordances recognition and discovery is
therefore as much tied to interaction with the object as it is to visual
stimuli. For example, in Sahin et al. \cite{Sahin:2007gr}, the authors
define an agent centric definition of affordances that relies of the
effects of interaction with the world.

Most of the affordance learning through interaction have so far been
focused on simple affordances such as pushing, rolling, etc. For example
in Modayil et. al \cite{Modayil:2008it} an autonomous robot learns about
the world through random motor babbling, observing changes in the
perceived world and makes goal oriented plans based on matching the
learned actions to a given plan. In Yuruten et al. \cite{Yuruten:2013hr}
the authors learn a robot to match simple descriptive adjectives and
nouns to object features using feature selection with a radial based
SVM.

Learning for more complex tasks such as interaction with everyday
objects is more complicated. In robotics the problem is usually
approached by Learning from Demonstration (LfD) as this is the way
humans also learns complex tasks. One common approach is to let the
robot imitate the demonstrator so as to reformulate the obsveration into
its own sensori-motor system.

One integral part in performing these complex tasks are grasping and
manipulation, where current research mostly have focused on grasping
objects regardless of task context \cite{Bohg:H95zG3Ya}. However, steps
have been taken towards learning task dependent grasp selection by
imitation as in Hjelm et. al \cite{Hjelm:2015hw}.

Understanding of how an object should be manipulated thus lays as much
in learning complex motor commands as recognizing how an object is best
manipulated in the currrent context. Realizing what in the feature
representation of an object makes the object afford the action is a key
part in the understanding of the context. From this understanding it
also follows that an agent can greatly reduce the space of possible
manipulation actions that it needs to consider, and allows it to gain a
deeper understanding of the affordance.

Previous efforts towards finding affordance bearing features have been
part in grasp planning models and focused on different forms of semantic
segmentation associated with possible grasping actions. In Aleotti et
al. \cite{Aleotti:2011hc} the authors segments objects as graphs where
each node can be linked to specific grasps. Similarly
\cite{Faria:2014ug, Hjelm:2014uc, Antanas:2014uo} segments the objects
main axis, and other functional areas, to find parts that can be
associated with particular types of grasps.

Since many affordances cannot be explained by local features our model
breaks with the locality assumptions of the grasp affordance learning
approaches by assuming that objects can both be formulated around both
global and local features. We also recognize that most man-made objects
within an affordance category have low variability and can be described
by simple features, such as the elongatedness of the object, the volume,
the specific color consistency across objects, or the local curvature.
By discovering the affordance bearing features across objects we have no
need for a tuned segmentation algorithm since our approach grounds the
affordance in the sensory input and in the case of local features
automatically induces a natural segmentation.

The approach most similar to ours is by Hermans et al.
\cite{Hermans:2011vz} and Sun \cite{JieSun:2010kv}. In Hermans the
authors learn a robot to recognize key attributes of objects such as
size, shape, color, material, and weight. Affordances are then
classified with the attributes as inputs and compared with a SVM trained
directly on the feature space. The direct approach performs comparably
or better, this is explained in terms of the feature space containing
information not directly explainable as any specific semantic attribute.
A similar conclusion can be reached from the results in
\cite{JieSun:2010kv}. We want to remark that if we follow the idea that
affordances should be grounded in the sensory input of the agent we can
conclude that imposing a human structure on top of how objects are
interpreted might not help the agent but rather confuse it; especially
when the individual features are extremely noisy. Language is a powerful
way to communicate and dissect knowledge however as their experiments
show what is right for one sensorimotor system might not be right for
another.

Like the afformentioned papers we also make use of discretization in the
creation of certain features but as it always means a trade-off between
discarding and keeping certain information we try to avoid it when
possible. This means that we let the algorithm make the choice of what
is needed for recognition. And contrary to
\cite{Hermans:2011vz, JieSun:2010kv} the information explaining the
affordance is not hidden in the object category or attributes but is
reflected in the transform and directly relatable to the input.

As mentioned above we are not interested in object categorization and
individuation as we think this is a hinderance to truly understanding
affordances. In humans there are cases where damage to the primary
visual cortex affected object recognition but not manipulation
\cite{Goodale:1991hc}. This suggests that discrimative features used in
recognition might be redunant in affordance recognition. We instead want
to formulate the understanding around the feature transform. This
removes the need for object category knowledge and enables the agent to,
in the case of a local feature, map the affordance to a specific portion
of the object. The transform can also help the agent to understand which
affordances are most similar by comparing the distances between the
transforms. Another benefit of learning the similarity measures is that
the human demonstrator can show the robot which objects are similar,
creating an intra category ordering, as opposed to it using a distance
measure to formulate the initial similarity.

\cite{Stark:2008bx}

\begin{comment}

\cite{Nikandrova:2015uu}
\cite{Thomaz:2009uk}
\cite{Griffith:2009cm}
ONE IDEA IS TO USE SEVERAL BOW REPRESENTATIONS AT DIFFERENT RESOLUTIONS!!



In the future we can expect that skill transfer will happen from robot to human, this will further the need for the agent in understandning for 

Skill-transfer from human to human works in the same manner as in the above experiment. We show some objects, explain the underlying invariant features, similarities and how the action works. The learner can then use the acquired skills to explore and act in the environment using the invariant knowledge to extrapolate and solve tasks. An important point is that that the transfer is not an exact copy due to the difference in semantic and sensory capabilities of the teacher and learner. 

In the future it is reasonable to assume that this skill-transfer will also happen between human and robots. Communication would be greatly enhanced by giving the robot the ability to ground the sensory input in semantic meaning, that is, find the the invariant features of the group and form metrics over these to discriminate which objects belongs to the group or not. Further on, utilizing this learning strategy as opposed to downloading an already trained classifier, would lead to greater adaptability and autonomy as the experience would be grounded in the robots own experience. 

If we go even further we can assume that in the future skill transfer will also happen from robot to human. Here semantic meaning grounded in sensory input  will be the only way to transfer knowledge as there is yet no known way of programming a human brain for learning complex tasks. 

Previous approaches to learning affordances have focused on....

Most of the above mentioned approaches focuses on learning affordances using large datasets that does not discover the invariant features of the object and where the focus has shifted from grounding to model accuracy. These models while useful for detection will not be very helpful in human-robot interaction. 




Lära eigen spectrumet dominanta plotta Renauds 

Hur annorlunda är representationen  för de olika tasken projektion av data punkter overlay. 
Vilka task är korrelerade? 

\end{comment}
